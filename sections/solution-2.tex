\begin{sol}{twodice-5}
    Let $\Omega=\{1,2,\ldots,6\}^2$, $A=\{(5,5)\}$, $B=\{(1,4),(2,3),(3,2),(4,1),(4,6),(5,5),(6,4)\}$. We are required to find $\PP(A\mid B)$. Note that $A\cap B=A$, hence
    \[
        \PP(A\mid B)=\frac{\PP(A)\cap\PP(B)}{\PP(B)}=\frac{\PP(A)}{\PP(B)}=\frac{1/36}{7/36}=\frac17.
    \]
\end{sol}

\begin{sol}{twodice-6-given-sum-8}
    Let $\Omega=\{1,2,\ldots,6\}^2$, and define the following events:
    \begin{gather*}
        A=\{(1,6),(2,6),(3,6),(4,6),(5,6),(6,6),(6,5),(6,4),(6,3),(6,2),(6,1)\}, \\
        B=\{(2,6),(3,5),(4,4),(5,3),(2,6)\}, \\
        A\cap B=\{(2,6),(6,2)\}.
    \end{gather*}
    Then
    \[
        \PP(A\mid B)=\frac{\PP(A\cap B)}{\PP(B)}=\frac{2/36}{5/36}=\frac25.
    \]
\end{sol}

\begin{sol}{threedice-distinct}
    We will work in $\Omega=\{1,2,\ldots,6\}^3$. Let $A$ be the event that at least one of the dice is showing $6$, and let $B$ be the event that all three dice are showing distinct numbers. Then $|B|=6\cdot5\cdot4=120$ and $|A\cap B|=\binom{5}{2}\cdot3!=10\cdot6=60$.
    \[
        \PP(A\mid B)=\frac{\PP(A\cap B)}{\PP(B)}=\frac{60/216}{120/216}=\frac{60}{120}=\frac12.
    \]
\end{sol}

\begin{sol}{prove-conditional}
    Multiplying both sides of $\frac{\PP(A\cap B)}{\PP(B)}=\PP(A\mid B)>\PP(A)$ by $\frac{\PP(B)}{\PP(A)}$ we get
    \[
        \PP(B\mid A)=\frac{\PP(A\cap B)}{\PP(A)}=\frac{\PP(B)}{\PP(A)}\frac{\PP(A\cap B)}{\PP(B)}>\frac{\PP(B)}{\PP(A)}\PP(A)=\PP(B).
    \]
\end{sol}

\begin{sol}{independent-subset}
    \[
        \PP(A)=\PP(A\cap B)=\PP(A)\PP(B)\quad\Longrightarrow\quad \PP(A)(1-\PP(B))=0.
    \]
    Therefore $\PP(A)=0$ or $\PP(B)=1$.
\end{sol}

\begin{sol}{independent-with-itself}
    Follows from the previous problem as a special case with $B=A$.
\end{sol}

\begin{sol}{bowl-probabilities}
    Let $W$, $B$, and $R$ be the events of both balls being white, black, and red, respectively. Each bowl contains $24$ balls, hence we have $\PP(W)=\frac{5\cdot10}{24^2}$, $\PP(B)=\frac{11\cdot8}{24^2}$, and $\PP(R)=\frac{8\cdot6}{24^2}$. These events are disjoint, therefore
    \[
        \PP(W\cup B\cup R)=\PP(W)+\PP(B)+\PP(R)=\frac{5\cdot10+11\cdot8+8\cdot6}{24^2}=\frac{186}{576}=\frac{31}{96}.
    \]
\end{sol}

\begin{sol}{pairwise-not-mutual}
    It's easy to see that $\PP(A_1\cap A_2) = \frac{3\cdot3}{6} = \frac36\cdot\frac36 = \PP(A_1)\cdot\PP(A_2)$ and $A_1\cap A_3 = A_2\cap A_3$, which implies that these events are pairwise independent, but not mutually independent, because $A_1\cap A_2\cap A_3=A_1\cap A_2$ (since even+odd is always odd), therefore
    \[
        \PP(A_1\cap A_2\cap A_3)=\PP(A_1\cap A_2)=\frac14\ne\frac12\cdot\frac12\cdot\frac12=\PP(A_1)\PP(A_2)\PP(A_3).
    \]
\end{sol}

\begin{sol}{not-all-same}
    If $A$ is the desired event, then $\PP(A)=1-\PP(\overline{A})=1-\left(\frac{\binom{3}{3}}{\binom{10}{3}}+\frac{\binom{5}{3}}{\binom{10}{3}}\right)=1-\left(\frac{1}{120}+\frac{11}{120}\right)=\frac{109}{120}$.
\end{sol}

\begin{sol}{two-shooters}
    We assume the events $A_1$ and $A_2$, of first and second shooters hitting the target, are independent. The probability that the target was hit at least once is then
    \begin{multline*}
        \PP(\text{hit at least once})=1-\PP(\text{both miss})=1-\PP(\overline{A_1}\cap\overline{A_2})=1-\PP(\overline{A_1})\PP(\overline{A_2})\\=1-0.3\cdot0.2=1-0.06=0.94,
    \end{multline*}
    and the probability of exactly one hit is
    \begin{multline*}
        \PP(\text{exactly one hit})=\PP\left((A_1\cap\overline{A_2})\cup(\overline{A_1}\cap A_2)\right)=\PP(A_1\cap\overline{A_2})+\PP(\overline{A_1}\cap A_2)\\=\PP(A_1)\PP(\overline{A_2})+\PP(\overline{A_1})\PP(A_2)=0.7\cdot0.2+0.3\cdot0.8=0.14+0.24=0.38.
    \end{multline*}
\end{sol}

\begin{sol}{five-points-circle-square}
    Let the radius of the circle be $r$. The diagonal of an inscribed square is $2r$, side length is $\sqrt{2}r$, and the area is $S_\square=(\sqrt{2}r)^2=2r^2$. The area of the circle is $S_\bigcirc=\pi r^2$, hence
    \[
        p=\PP(\text{point in square})=\frac{2r^2}{\pi r^2}=\frac{2}{\pi}.
    \]
    The region outside the square is composed of 4 segments of equal area, that add up to $\pi r^2-2r^2$, so each one has area $\frac{\pi r^2-2r^2}{4}$. Therefore
    \[
        q=\PP(\text{point in a chosen segment})=\frac{\pi r^2-2r^2}{4\pi r^2}=\frac{\pi-2}{4\pi}.
    \]
    The desired probability is
    \[
        \binom{5}{1}\cdot p\cdot 4!\cdot q^4 = 120\cdot\frac{2}{\pi}\left(\frac{\pi-2}{4\pi}\right)^4.
    \]
\end{sol}

\begin{sol}{student-library}
    Since the probability that the student will find the book in a library is $0.5$, and the probability that the book will be occupied is $0.5$, the probability of the student obtaining the book at a library is $0.25$. Hence the probability that the student will not get the book in a library is $1-0.25=0.75$. Because the libraries are independently operated, the probability that the student does not obtain the book after visiting all three libraries is $(0.75)^3=\frac{27}{64}\approx0.42$. Therefore it is more likely that the student will get the book.
\end{sol}