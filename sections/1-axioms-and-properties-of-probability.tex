\newpage
\section{Axioms and Properties of Probability}

\fbox{\begin{minipage}{0.987602233886\linewidth}

\begin{definition}[$\sigma$-algebra]
    Let $\Omega$ be a nonempty set. A nonempty subset $\mathcal{F}$ of $2^\Omega$ is called a \textit{$\sigma$-algebra} over $\Omega$, if the following two conditions are satisfied:
    \begin{enumerate}
        \item If $A\in\mathcal{F}$ then $\overline{A}\in\mathcal{F}$, where $\overline{A}=\Omega\setminus A$.
        \item If $A_1,A_2,\ldots\in\mathcal{F}$ then $\bigcap_{i=1}^{\infty}A_i\in\mathcal{F}$.
    \end{enumerate}
\end{definition}

\begin{definition}[probability]
    Let $\mathcal{F}$ be a $\sigma$-algebra. A function $\mathbb{P}:\mathcal{F}\rightarrow\mathbb{R}$ is called a \textit{probability measure} on $\mathcal{F}$ (or just \textit{probability}) if the following three conditions are satisfied:
    \begin{enumerate}
        \item $\mathbb{P}(A)\ge0$ for all $A\in\mathcal{F}$.
        \item $\mathbb{P}(\Omega)=1$.
        \item If $A_1,A_2,\ldots\in\mathcal{F}$ are pairwise disjoint ($A_i\cap A_j=\varnothing$ for all $i\ne j$) then
        \[
            \mathbb{P}\left(\bigcup_{i=1}^{\infty}A_i\right) = \sum_{i=1}^{\infty}\mathbb{P}(A_i).
        \]
    \end{enumerate}
\end{definition}

If $\mathcal{F}$ is a $\sigma$-algebra over $\Omega$ then the ordered pair $(\Omega,\mathcal{F})$ is called an \textit{experiment} and if in addition $\mathbb{P}$ is a probability measure on $\mathcal{F}$ then the ordered triple $(\Omega,\mathcal{F},\mathbb{P})$ is called a \textit{probability space}. The set $\Omega$ is called the \textit{sample space} of the experiment and the elements of $\Omega$ are called \textit{elementary outcomes} or just \textit{outcomes}. The elements of $\mathcal{F}$ are called \textit{events}.

\end{minipage}}

\begin{enumerate}[series=prob]
    \prob{empty-probability}{Prove that the probability of the empty set is $0$.}
    \prob{nonempty-sample-space}{Explain why we require the sample space $\Omega$ to be nonempty.}
    \prob{coin1}{A fair coin is tossed three times. Model the experiment and describe the sample space. Find the probability of getting exactly two heads.}
    \prob{twodie1}{Two fair six-sided dice are rolled. Describe the probability space. Find the probability that their sum is $7$.}
    \prob{monotonicity}{Prove the monotonicity of probabilty: $A\subseteq B\;\Longrightarrow\;\mathbb{P}(A)\le\mathbb{P}(B)$ for all $A,B\in\mathcal{F}$.}
    \prob{cointwoconsec53}{
        Consider the following experiment: A coin is tossed until we get two consecutive Tails. Describe the following events:
        \begin{enumerate}
            \item $A$ - The number of tosses is 5.
            \item $B$ - Heads came up at most 3 times.
        \end{enumerate}
    }
    \prob{finiteadd}{
        Prove the finite additivity of probability: If the events $A_1,A_2,\ldots,A_n\in\mathcal{F}$ are pairwise disjoint, then
        \[
            \mathbb{P}\left(\bigcup_{i=1}^{n}A_i\right) = \sum_{i=1}^{n}\mathbb{P}(A_i).
        \]
    }
    \prob{shopofsamplespaces}{
        The shop of sample spaces is open from 10:00 to 20:00. A student enters and leaves the shop at $x$ and $y$ hours, respectively. Describe the sample space $\Omega$ using elementary outcomes as $(x,y)$ pairs. Use the geometric probability model, draw $\Omega$ and the following events:
        \begin{enumerate}
            \item The student was in the shop for no more than 2 hours.
            \item The student was in the shop at $15:00$.
            \item The student was in the shop for the entire period from $14:00$ to $16:00$.
            \item The student was in the shop at some point between $14:00$ and $16:00$.
            \item The student's entering time is closer to the opening time, than leaving time to the closing time.
        \end{enumerate}
    }
    \prob{shopofsamplespaces2}{Find the probabilities of each event in the previous problem.}
    \prob{complement-probability}{Prove that $\mathbb{P}(\overline{A})=1-\mathbb{P}(A)$ for all $A\in\mathcal{F}$.}
    \prob{shootingthreetimes}{
        Consider the experiment of shooting at a target three times. Let $A_i$ ($i=1,2,3$) be the event of $i$-th bullet hitting the target. Describe the following events using the events $A_i$:
        \begin{enumerate}
            \item The target was hit all three times.
            \item The target was hit at least twice.
            \item The target was hit exactly twice.
            \item the target was hit at least once.
            \item The target was hit exactly once.
            \item The target was not hit.
            \item The target was hit an even number of times.
            \item The target was hit an odd number of times.
        \end{enumerate}
    }
    \prob{geom1}{Let $A=\{(x,y)\in\Omega\;\vert\; x+y<1\}$ and $B=\{(x,y)\in\Omega\;\vert\;y\le x^2\}$ be events of the sample space $\Omega=[0,1]\times[0,1]$. Draw on the plane the events $A\cup B$, $A\cap B$, $A\setminus B$ $\overline{A}\cup\overline{B}$. Find the probabilities of each of these events in the geometric probability model.}
    \prob{polynom1}{
        Consider the equation $ax^2+bx+c=0$. Let the coefficients $a,b,c$ be decided as the result of throwing a fair die three times. Find the probability that
        \begin{enumerate}
            \item the roots of the equation are real.
            \item the roots of the equation are rational.
        \end{enumerate}
    }
    \prob{inclusion-exclusion-simple}{Prove that $\mathbb{P}(A\cup B) = \mathbb{P}(A)+\mathbb{P}(B)-\mathbb{P}(A\cap B)$.}
    \prob{weirdlimit}{Find the probability $p_n$ that a randomly chosen number from the set $\{1,2,\ldots,n\}$ is divisible by a fixed positive integer $k$. Find $\displaystyle\lim_{n\rightarrow\infty}p_n$.}
    \prob{union-bound}{
        Prove the union bound (Boole's inequality):
        \[
            \mathbb{P}\left(\bigcup_{i=1}^{\infty}A_i\right)\le\sum_{i=1}^{\infty}\mathbb{P}(A_i).
        \]
    }
    \prob{unit-circle-square}{A point is chosen randomly inside the unit circle. Find the probability that the chosen point is inside a fixed square inscribed in the circle. Use the geometric probability model.}
    \prob{geometric4d}{Two points are randomly chosen in $\Omega=[0,1]$. Find the probability that the distance between those points is less than $\alpha\in(0,1)$. Use the geometric probability model.}
    \prob{polynom2}{
        Find the probability that the roots of the equation $x^2+ax+b=0$ are
        \begin{enumerate}
            \item real,
            \item positive,
        \end{enumerate}
        if the coefficients $0\le a\le 1$ and $0\le b\le 1$ are chosen randomly. Use the geometric probability model.
    }
    \prob{polynom3}{
        Find the probability that the roots of the equation $x^2+2ax+b=0$ are
        \begin{enumerate}
            \item real,
            \item positive,
        \end{enumerate}
        if the coefficients are chosen randomly and $|a|\le1$, $|b|\le1$.
    }
    \prob{incl-excl-proof}{
        Prove the inclusion-exclusion principle for probability:
        \[
            \mathbb{P}\left(\bigcup_{i=1}^{n}A_i\right) = \sum_{i=1}^{n}\left((-1)^{i-1}\sum_{k_1<\ldots<k_i} \bigcup_{l=1}^{i}A_l\right)
        \]
    }
    \prob{stick-break-triangle}{
        A stick of length 1 is randomly broken into three pieces. Find the probability that those three pieces can form a triangle. Use the geometric probability model.
    }
    \prob{continuity}{
        Prove the continuity of probability for increasing sequences: if $A_1\subseteq A_2\subseteq\ldots$ then
        \[
            \mathbb{P}\left(\bigcup_{i=1}^{\infty}A_i\right) = \lim_{n\rightarrow\infty}\mathbb{P}(A_n).
        \]
    }
    \prob{five-envelopes}{There are 5 different letters and 5 different envelopes. Each letter is placed in an envelope at random. What is the probability that no letter is placed in its correct envelope?}
    \prob{polynom4}{Find the probability of the roots of the equation $ax^2+bx+c=0$ being real if the coefficients are chosen randomly in the cube $0<a\le1$, $0<b\le1$, $0<c\le1$. Use the geometric probability model.}
    \prob{nonmeasurable}{Prove that it is not possible to define a probability on the experiment $([0,1],2^{[0,1]})$ such that the probability of any interval is its length.}
\end{enumerate}