\begin{sol}{empty-probability}
    Let $A_n=\varnothing$ for all positive integers $n$ and denote. Then we can use the third axiom (condition) of probability on these sets as they are pairwise disjoint ($\varnothing\cap\varnothing=\varnothing$) to get
    \[
        \PP(\varnothing)=\PP\left(\bigcup_{i=1}^{\infty}A_i\right)=\sum_{i=1}^{\infty}\PP(A_i)=\sum_{i=1}^{\infty}\PP(\varnothing),
    \]
    therefore $\PP(\varnothing)=0$, because otherwise the sum $\sum_{i=1}^{\infty}\PP(\varnothing)$ would diverge to infinity.
\end{sol}

\begin{sol}{nonempty-sample-space}
    According to the previous exercise $\PP(\varnothing)=0$, which would contradict the second axiom of probability, namely $\PP(\Omega)=1$, if $\Omega=\varnothing$.
\end{sol}

\begin{sol}{coin1}
    We can use the sample space $\Omega=\{H,T\}^3=\{(H,H,H),(H,H,T),\ldots,(T,T,T)\}$, where each elementary outcome has probability $\frac18$. The probability of getting exactly two heads is the probability of the event $\{(H,H,T),(H,T,H),(T,H,H)\}$, which is $\frac38$.
\end{sol}

\begin{sol}{twodie1}
    One way of modeling this experiment is to use the probability space $(\Omega,\mathcal{F},\PP)$, where $\Omega=\{1,2,\ldots,6\}^2=\{(1,1),(1,2),\ldots,(6,6)\}$, $\mathcal{F}=2^\Omega$, and $\PP(A)=\frac{|A|}{|\Omega|}=\frac{|A|}{36}$ for $A\in\mathcal{F}$. Let $B$ be the event of the sum being $7$. Then $B=\{(1,6),(2,5),(3,4),(4,3),(5,2),(6,1)\}$, and therefore $\PP(B)=\frac{|B|}{36}=\frac{6}{36}=\frac{1}{6}$.
\end{sol}

\begin{sol}{monotonicity}
    Set $A_1=A$, $A_2=B\setminus A$, and $A_i=\varnothing$ for all integers $i\ge3$. Then the events $A_i$ are pairwise disjoint and $\bigcup_{i=1}^{\infty}A_i=A\cup(B\setminus A)=B$, so it follows from the third axiom of probability that
    \[
        \PP(B)=\PP\left(\bigcup_{i=1}^{\infty}A_i\right)=\sum_{i=1}^{\infty}\PP(A_i)=\PP(A)+\PP(B\setminus A)+0+0+\cdots=\PP(A)+\PP(B\setminus A),
    \]
    which means that $\PP(B)\ge\PP(A)$ because $\PP(B\setminus A)\ge0$ according to the first axiom of probability.
\end{sol}

\begin{sol}{cointwoconsec53}
    Let $\Omega=\{TT,HTT,THTT,HHTT,HTHTT,THHTT,HHHTT,\ldots\}$. Here
    \begin{align*}
        A=\{&HTHTT,THHTT,HHHTT\},\\
        B=\{&TT,HTT,THTT,HHTT,HTHTT,THHTT,\\&HHHTT,HHTHTT,HTHHTT,THHHTT\}.
    \end{align*}
\end{sol}

\begin{sol}{finiteadd}
    Define $A_i=\varnothing$ for all integers $i\ge n+1$. Then using the third axiom of probability we get
    \[
        \PP\left(\bigcup_{i=1}^{n}A_i\right)=\PP\left(\bigcup_{i=1}^{\infty}A_i\right)=\sum_{i=1}^{\infty}\PP(A_i)=\sum_{i=1}^{n}\PP(A_i).
    \]
\end{sol}

\begin{sol}{complement-probability}
    We have already established finite additivity in Problem~\ref{prob:finiteadd}, which impllies that
    \[
        \PP(A)+\PP(\overline{A})=\PP(A\cup\overline{A})=\PP(\Omega)=1,
    \]
    hence $\PP(\overline{A})=1-\PP(A)$.
\end{sol}